% Dokumentenklasse und Seitenformat
\documentclass[a4paper,BCOR=12mm]{scrartcl}  % BCOR für Bindungskorrektur hinzugefügt
\usepackage[top=2cm, bottom=2cm, left=2cm, right=2cm]{geometry}

% Schriftarten und Typografie
\usepackage{fontspec}  % Ermöglicht die Verwendung von TrueType/OpenType-Schriftarten mit XeLaTeX
\usepackage[T1]{fontenc} % Korrekte Darstellung von Umlauten und Sonderzeichen
\usepackage{textcomp}   % Zusätzliche Schriftzeichen wie Euro-Symbol
\usepackage{microtype}  % Optimiert die Mikrotypografie für bessere Lesbarkeit

% Hauptschriftart einstellen (Tex Gyre Termes, aber anpassbar)
\setmainfont[
    Path = /Users/jan/Library/Fonts/,  % Pfad zu den Schriftarten
    BoldFont = texgyretermes-bold.otf,
    ItalicFont = texgyretermes-italic.otf,
    BoldItalicFont = texgyretermes-bolditalic.otf
]{texgyretermes-regular.otf}

% Monospaced-Schriftart für Quellcode oder ähnliches
\setmonofont{Source Code Pro}

% Mathematische Schriftarten und Pakete
\usepackage{unicode-math} % Verwendung moderner Mathematik-Schriften (kompatibel mit XeLaTeX)
\setmathfont{STIX Two Math} % Mathematische Schriftart einstellen

% Mathematische Erweiterungen
\usepackage{amsmath}  % Erweiterungen für mathematische Formeln
%\usepackage{amssymb}  % Weitere mathematische Symbole
\usepackage{siunitx}  % Für das Setzen von SI-Einheiten und Zahlenformaten
\sisetup{per-mode=symbol}  % Optional: Ändert die Schreibweise von Einheiten
\sisetup{locale = DE}
% Definiere zusätzliche Einheiten
\DeclareSIUnit{\litre}{l}
\DeclareSIUnit{\PS}{PS}


% Sprache und Lokalisierung (Deutsch)
\usepackage[ngerman]{babel}   % Deutsche Sprachunterstützung
\usepackage{csquotes}         % Kontext-sensitive Anführungszeichen (wichtig für Deutsch und Zitate)

% Zusätzliche Pakete für Symbole, Vektoren und Grafiken
\usepackage{pifont}     % Für Sonderzeichen wie Häkchen oder Kreuze
\usepackage{esvect}     % Für Vektor-Symbole
\usepackage{graphicx}   % Grafiken einfügen
\usepackage{caption}    % Anpassung von Bildunterschriften
\usepackage{subcaption} % Mehrere Unterbilder
\usepackage{tikz}
\usetikzlibrary{mindmap, shapes, backgrounds}

% Tabellen und Listen
\usepackage{booktabs}   % Hochwertige Tabellen
\usepackage{multirow}
\usepackage{makecell}
\usepackage{tabularx}   % Flexible Tabellen
\usepackage{enumitem}   % Anpassen von Aufzählungen und Listen
\setlist[enumerate,1]{label=\alph*)}
\setlist{itemsep=0pt}   % Kein Abstand zwischen Listenelementen

% Algorithmen
\usepackage[ruled,vlined]{algorithm2e} % Für die Darstellung von Algorithmen
\SetAlCapNameFnt{\usekomafont{caption}} % Anpassung der Algorithmentitel
\SetAlCapFnt{\usekomafont{captionlabel}} % Anpassung der Algorithmus-Kopfzeile

% Abstract-Anpassung
\usepackage{abstract} % Anpassung der Zusammenfassung

% Literaturverzeichnis
\usepackage[backend=biber, style=ieee, defernumbers=true]{biblatex}  % Verwendung von Biber als Backend und IEEE-Stil (kompatibel mit XeLaTeX)
\addbibresource{references.bib}  % Einbinden der Bibliographie-Datei
\ExecuteBibliographyOptions{backref=true,backrefstyle=three+,url=true,urldate=comp,abbreviate=false,maxbibnames=20}
\DeclareBibliographyCategory{cited}
\let\defaultcite\cite
\renewcommand*\cite[2][]{\addtocategory{cited}{#2}\defaultcite[#1]{#2}}

% Farben und Listings für Quellcode
\usepackage[dvipsnames,svgnames,x11names]{xcolor} % Erweiterte Farbunterstützung
% colors_settings.tex
\providecommand{\definecolor}{}
\definecolor{DodgerBlue4}{rgb}{0.06, 0.31, 0.55}
\definecolor{DarkOrange}{rgb}{1.0, 0.55, 0}
\definecolor{DarkSlateGray}{rgb}{0.18,0.31,0.31}
\definecolor{gray33}{rgb}{0.33,0.33,0.33}
\usepackage{listings}
% listing_settings.tex
\lstset{
    basicstyle=\ttfamily\small,  % Monospaced-Schriftart für Quellcode
    language=C++,  % Programmiersprache
    breaklines=true,  % Zeilenumbruch
    showspaces=false,  % Keine Leerzeichen anzeigen
    showstringspaces=false,  % Keine Leerzeichen in Strings anzeigen
    showtabs=false,  % Keine Tabs anzeigen
    tabsize=4,  % Tabulatorgröße
    captionpos=t,  % Position der Bildunterschrift
    breakatwhitespace=false,  % Zeilenumbruch bei Leerzeichen
    title=\lstname,  % Titel des Listings
    keywordstyle=\color{DodgerBlue4},  % Stil für Schlüsselwörter
    commentstyle=\color{gray33},  % Stil für Kommentare
    stringstyle=\color{DarkOrange},  % Stil für Strings
    %morekeywords={std, cout, endl},  % Zusätzliche Schlüsselwörter
    identifierstyle=\bfseries\color{black},  % Stil für Identifikatoren
    floatplacement=htbp,
    abovecaptionskip=.5\baselineskip,
    belowcaptionskip=.5\baselineskip,
    upquote=true,
    literate={á}{{\'a}}1 {é}{{\'e}}1 {í}{{\'i}}1 {ó}{{\'o}}1 {ú}{{\'u}}1
                {Á}{{\'A}}1 {É}{{\'E}}1 {Í}{{\'I}}1 {Ó}{{\'O}}1 {Ú}{{\'U}}1
                {à}{{\`a}}1 {è}{{\`e}}1 {ì}{{\`i}}1 {ò}{{\`o}}1 {ù}{{\`u}}1
                {À}{{\`A}}1 {È}{{\'E}}1 {Ì}{{\`I}}1 {Ò}{{\`O}}1 {Ù}{{\`U}}1
                {ä}{{\"a}}1 {ë}{{\"e}}1 {ï}{{\"i}}1 {ö}{{\"o}}1 {ü}{{\"u}}1
                {Ä}{{\"A}}1 {Ë}{{\"E}}1 {Ï}{{\"I}}1 {Ö}{{\"O}}1 {Ü}{{\"U}}1
                {â}{{\^a}}1 {ê}{{\^e}}1 {î}{{\^i}}1 {ô}{{\^o}}1 {û}{{\^u}}1
                {Â}{{\^A}}1 {Ê}{{\^E}}1 {Î}{{\^I}}1 {Ô}{{\^O}}1 {Û}{{\^U}}1
                {œ}{{\oe}}1 {Œ}{{\OE}}1 {æ}{{\ae}}1 {Æ}{{\AE}}1 {ß}{{\ss}}1
                {ű}{{\H{u}}}1 {Ű}{{\H{U}}}1 {ő}{{\H{o}}}1 {Ő}{{\H{O}}}1
                {ç}{{\c c}}1 {Ç}{{\c C}}1 {ø}{{\o}}1 {å}{{\r a}}1 {Å}{{\r A}}1
                {€}{{\EUR}}1 {£}{{\pounds}}1 {~}{{\textasciitilde}}1 {-}{{-}}1
}
    

% Hyperlinks und Querverweise
\usepackage[hidelinks]{hyperref}  % Versteckte Links (ohne Umrandung)
\usepackage[ngerman]{cleveref}    % Intelligente Querverweise

% Fußnoten
\usepackage{footnote}   % Verbesserte Fußnotenverwaltung
\usepackage{fnpct}      % Verwaltung der Fußnoten und Interaktion mit Satzzeichen
\setfnpct{after-punct-space={-.2em}}

% Kopf- und Fußzeilen
\usepackage[automark]{scrlayer-scrpage} 
\pagestyle{scrheadings}

% Einstellungen für Kopf- und Fußzeilen
\clearpairofpagestyles % Alle Voreinstellungen löschen
\ihead[]{\headmark} % Kapitel oder Abschnittstitel in die Kopfzeile
\ofoot*[\pagemark]{\pagemark} % Seitenzahl außen in die Fußzeile

\automark[section]{section} 
\automark*[subsection]{}

% Benutzerdefinierte Befehle
\newcommand{\R}{\mathbb{R}}     % Für die Menge der reellen Zahlen
\newcommand{\square}{\ding{111}} % Für Checkboxen
\renewcommand{\lstlistingname}{Quelltext}

% Dokumenteninformationen
\title{Thema \\
       Notizen}
\author{}
\date{Stand: \today}

\begin{document}

\maketitle

\begin{abstract}
    Zsf.
\end{abstract}
    
\tableofcontents

% Beginn des eigentlichen LaTeX-Dokuments
%%%%%%%%%%%%%%%%%%%%%%%%%%%%%%%%%%%%%%%%%%%%%%%%%%%%%%%%%%%%%%%%%%%%%%%%%%%%%%%%
\section{To-Do-Liste}
\begin{itemize}[label=\ding{113}] % leeres Quadrat
    \item Aufgabe 1
    \item Aufgabe 2
    \item Aufgabe 3
\end{itemize}

\section*{C++ Hallo Welt}

\begin{lstlisting}[caption={Hallo Welt in C++}]
#include <iostream>

int main() {
    std::cout << "Hallo Welt!" << std::endl;
    return 0;
}
\end{lstlisting}


\section*{Algorithmus}

% Beispiel für einen strukturierten Algorithmus
\begin{algorithm}[H]
\caption{Maximalwert in einem Array finden}
\KwIn{Ein Array $A[1 \dots n]$ mit $n$ Elementen}
\KwOut{Der Maximalwert in $A$}
\KwData{Ein Array von Zahlen}
\KwResult{Größtes Element im Array}

\textbf{Initialisierung:} Setze $max = A[1]$\;
\For{$i = 2$ \textbf{bis} $n$}{
    \If{$A[i] > max$}{
        Setze $max = A[i]$\;
    }
}
\textbf{Ausgabe:} Gib $max$ zurück\;

\end{algorithm}

\section{\LaTeX{} Cheatsheet für normgerechte Einheitenschreibweise}

\begin{enumerate}
    \item Laden Sie das \texttt{siunitx}-Paket:
    \begin{verbatim}
    \usepackage{siunitx}
    \sisetup{per-mode=symbol}
    \sisetup{locale = DE}
    % Definiere zusätzliche Einheiten
    \DeclareSIUnit{\litre}{l}
    \DeclareSIUnit{\PS}{PS}
    \end{verbatim}

    \item Grundlegende Verwendung:
    \begin{verbatim}
    \SI{3}{\kilo\metre}  % Für 3 km
    \SI{5}{\kilo\watt\hour}  % Für 5 kWh
    \end{verbatim}
    Beispiel: \SI{3}{\kilo\metre}, \SI{5}{\kilo\watt\hour}

    \item Komplexe Einheiten:
    \begin{verbatim}
    \SI{287}{\joule\per (\kilogram \cdot \kelvin)}
    \end{verbatim}
    Beispiel: \SI{287}{\joule\per (\kilogram \cdot \kelvin)}

    \item Einheiten ohne Wert:
    \begin{verbatim}
    \si{\metre\per\second}  % Für m/s
    \end{verbatim}
    Beispiel: \si{\metre\per\second}

    \item Benutzerdefinierte Einheiten:
    \begin{verbatim}
    \DeclareSIUnit{\PS}{PS}  % Definiert PS als Einheit
    \SI{150}{\PS}  % Verwendet PS in einer Messung
    \end{verbatim}
    Beispiel: \SI{150}{\PS}

    \item Achsenbeschriftungen:
    \begin{verbatim}
    \sisetup{per-mode=symbol}
    $s/\si{\kilo\metre}$  % Für s/km
    \end{verbatim}
    Beispiel: $s/\si{\kilo\metre}$

    \item Brüche in Einheiten:
    \begin{verbatim}
    \SI{5}{\litre\per (100\, \kilo\metre)}  % Für 5 l/(100 km)
    \end{verbatim}
    Beispiel: \SI{5}{\litre\per (100\, \kilo\metre)}

    \item Kursive Formelzeichen mit normalen Einheiten:
    \begin{verbatim}
    $\mathit{v} = \SI{100}{\kilo\metre\per\hour}$
    \end{verbatim}
    Beispiel: $\mathit{v} = \SI{100}{\kilo\metre\per\hour}$

    \item Komplexes Beispiel:
    \begin{verbatim}
    \mathit{V_S} = \SI{5}{\litre\per (100\, \kilo\metre)}
    \end{verbatim}
    Beispiel: $\mathit{V_S} = \SI{5}{\litre\per (100\, \kilo\metre)}$
\end{enumerate}
%%%%%%%%%%%%%%%%%%%%%%%%%%%%%%%%%%%%%%%%%%%%%%%%%%%%%%%%%%%%%%%%%%%%%%%%%%%%%%%%

\section{Links}

% Beispiel für eine Fußnote mit URL
Google\footnote{\url{https://www.google.com}}.

% Beschreibung von Claude AI
Claude\footnote{\url{https://claude.ai/new}} ist ein KI-Assistent, der von Anthropic (San Francisco, Kalifornien, USA) entwickelt wurde. Er ist darauf ausgelegt, bei einer Vielzahl von Aufgaben zu unterstützen, darunter Analyse, Problemlösung, Programmierung und kreatives Schreiben.

% Beispiel für eine Zitation
OpenAI \cite{statista2024openai} ist ein führendes KI-Unternehmen, das für Anwendungen wie ChatGPT\footnote{\url{https://chatgpt.com/?model=gpt-4}} und Dall-E bekannt ist.

% Ausgabe des Literaturverzeichnisses
\cleardoublepage % Startet auf einer neuen rechten Seite
\phantomsection % Ermöglicht es, dass das Inhaltsverzeichnis den richtigen Link enthält
\addcontentsline{toc}{section}{Literatur} % Fügt "Literatur" dem Inhaltsverzeichnis als Abschnitt hinzu
\printbibliography % Zeigt alle zitierten Einträge an


\end{document}